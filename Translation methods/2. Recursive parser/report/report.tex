% !TEX encoding = UTF-8
% !TEX program = pdflatex
\documentclass[12pt]{article}

\usepackage[T2A]{fontenc}
\usepackage[utf8]{inputenc}
\usepackage[russian]{babel}
\textheight=24cm
\textwidth=16cm
\oddsidemargin=0pt
\topmargin=-1.5cm

\title{Лабораторная работа №2}
\author{\copyright~~Плотников Андрей}
\date{13 декабря 2017 г.}
\begin{document}
\maketitle
\thispagestyle{empty}
Вариант 6. Визуализация переменных в Си

Описания переменных в Си. Сначала следует имя типа, затем
разделенные  запятой  имена  переменных.  Переменная  может  быть
указателем, в этом случае перед ней идет звездочка (возможны и
указатели на указатели, и т. д.). Описаний может быть несколько.
Используйте один терминал для всех имен переменных и имен типов.

Пример

    \begin{verbatim}
        int a, *b, ***c, d;
    \end{verbatim}

\begin{itemize}
    \item \textbf{Разработка грамматики}

            Построим грамматику.
            
            S -- стартовый нетерминал

            D -- объявление переменной

            T -- тип переменных

            K -- разделитель типа и имени

            F -- первая переменная

            N -- последующие переменные

            V -- имя переменной
            

            S $\rightarrow$ D

            D $\rightarrow$ TKD

            D $\rightarrow$ $\varepsilon$

            K $\rightarrow$ ' 'F

            F $\rightarrow$ VN

            N $\rightarrow$ ,VN

            N $\rightarrow$ ;

            V $\rightarrow$ *V

            V $\rightarrow$ a

            T $\rightarrow$ a


            В грамматике нет левой рекурсии.

    \item \textbf{Построение лексического анализатора}
        \begin{verbatim}
private enum Token {
    ASTERISK,  // *
    COMMA,     // ,
    SEMICOLON, // ;
    NAME,      // a-z
    END        // $
}

public class Parser {
    
    private static String processingString;
    private static Token processingToken;
    private static char proccessingChar;
    private static int offset;
    
    public static Node parse (String input) {
        processingString = input + "$";
        processingToken = null;
        proccessingChar = 0;
        offset = 0;
        
        nextToken ();
        return fetchEnterPoint ();
    }
    
    private static boolean isBlank (char symbol) {
        return Character.isWhitespace (symbol);
    }
    
    private static boolean isControl (char symbol) {
        return symbol == ',' || symbol == ';';
    }
    
    private static boolean hasMore () {
        return offset < processingString.length ();
    }
    
    private static void nextChar () {
        proccessingChar = processingString.charAt (offset ++);
    }
    
    private static void nextToken () {
        while (hasMore () && isBlank (proccessingChar)) {
            nextChar (); // Skipping blank spaces
        }
        
        if (isBlank (proccessingChar)) {
            throw new ParseException ("End of string reached at ", offset);
        }
        
        switch (proccessingChar) {
            case '*':
                processingToken = Token.ASTERISK;
                nextChar ();
                break;
            case ',':
                processingToken = Token.COMMA;
                nextChar ();
                break;
            case ';':
                processingToken = Token.SEMICOLON;
                nextChar ();
                break;
            case '$':
                processingToken = Token.END;
                break;
            
            default:
                while (hasMore () 
                        && !isBlank (proccessingChar)
                        && !isControl (proccessingChar)) {
                    nextChar ();
                }
                
                processingToken = Token.NAME;
        }
    }
    
    ...
}
        \end{verbatim}

    \item \textbf{Построение синтаксического анализатора}

        Построим множества FIRST и FOLLOW для нетерминалов грамматики

        \begin{tabular} {|c|c|c|}
            \hline
                \textbf{Нетерминал} & \textbf{FIRST} & \textbf{FOLLOW} \\
            \hline
                S & $\varepsilon$,  a & \$ \\
            \hline
                D & $\varepsilon$, a & \$ \\
            \hline
                K & space & a \\
            \hline
                F & *, a & a \\
            \hline
                N & \textbf{,}, ; & a \\
            \hline
                V & *, a & \textbf{,}, ; \\
            \hline
                T & a & space \\
            \hline
        \end{tabular}


        Структура данных для хранения дерева
        \begin{verbatim}
public class Node {

    private final List <Node> _list;
    private final String _name;
    
    public Node (String name, Node... children) {
        this._list = new ArrayList <> (Arrays.asList (children));
        this._name = name;
    }
    
}
        \end{verbatim}

        Синтаксический анализатор

        \begin{verbatim}
public class Parser {
    ...

    // S
    private static Node fetchEnterPoint () {
        String method = "Enter point";
        switch (processingToken) {
            case NAME:
                return new Node (method, 
                                    fetchVariableDeclaration ());
            case END:
                return new Node (method, new Node ("$"));
                
            default:
                throw new ParseException ("Enter point not found at ", 
                                            offset);
        }
    }
    
    // D
    private static Node fetchVariableDeclaration () {
        String method = "Variable declaration";
        switch (processingToken) {
            case NAME:
                Node type = fetchVariableType ();
                Node skip = fetchVariableSkiper ();
                return new Node (method, type, skip, 
                                    fetchVariableDeclaration ());
            case END:
                return new Node (method, new Node ("$"));
                
            default:
                throw 
                    new ParseException ("Variable declaration not found at ", 
                                            offset);
        }
    }
    
    // T
    private static Node fetchVariableType () {
        String method = "Variable type";
        switch (processingToken) {
            case NAME:
                nextToken ();
                return new Node (method, new Node ("a"));
                
            default:
                throw new ParseException ("Variable type not found at ", 
                                            offset);
        }
    }
    
    // K
    private static Node fetchVariableSkiper () {
        String method = "Variable skiper";
        switch (processingToken) {
            case ASTERISK:
            case NAME:
                return new Node (method, fetchVariableEnterance ());
                
            default:
                throw new ParseException ("Variable skiper not found at ", 
                                            offset);
        }
    }
    
    // F
    private static Node fetchVariableEnterance () {
        String method = "Variable enterance";
        switch (processingToken) {
            case ASTERISK:
            case NAME:
                Node variable = fetchVariable ();
                Node next = fetchNextVariable ();
                return new Node (method, variable, next);
                
            default:
                throw 
                    new ParseException ("Variable enterance not found at ", 
                                            offset);
        }
    }
    
    // V
    private static Node fetchVariable () {
        String method = "Variable";
        switch (processingToken) {
            case ASTERISK:
                nextToken ();
                Node variable = fetchVariable ();
                return new Node (method, new Node ("*"), 
                                    variable);
            case NAME:
                nextToken ();
                return new Node (method, new Node ("a"));
                
            default:
                throw new ParseException ("Variable not found at ", 
                                            offset);
        }
    }
    
    // N
    private static Node fetchNextVariable () {
        String method = "Next variable";
        switch (processingToken) {
            case COMMA:
                nextToken ();
                Node variable = fetchVariable ();
                Node next = fetchNextVariable ();
                return new Node (method, new Node (","), 
                                    variable, next);
            case SEMICOLON:
                nextToken ();
                return new Node (method);
                
            default:
                throw new ParseException ("Variable not found at ", 
                                            offset);
        }
    }
}
        \end{verbatim}				



    \item \textbf{Визуализация дерева разбора}

        В качестве примера для визуализации возьмём следующий тест:

        \begin{verbatim}				
int a, *b, **c; double n, m;
        \end{verbatim}				

        Дерево разбора
        \begin{verbatim}				
Enter point (S)
|- Variable declaration (D)
    |- Variable type (T)
        |- a
    |- Variable skiper (K)
        |- Variable enterance (F)
            |- Variable (V)
                |- a
            |- Next variable (N)
                |- ,
                |- Variable (V)
                    |- *
                    |- Variable (V)
                        |- a
            |- Next variable (N)
                |- ,
                |- Variable (V)
                    |- *
                    |- Variable (V)
                        |- *
                        |- Variable (V)
                            |- a
                |- Next variable (N)
    |- Variable declaration (D)
        |- Variable type (T)
            |- a
        |- Variable skiper (K)
            |- Variable enterance (F)
                |- Variable (V)
                    |- a
            |- Next variable (N)
                |- ,
                |- Variable (V)
                    |- a
                |- Next variable (N)
        |- Variable declaration (D)
            |- $
        \end{verbatim}				

    \item \textbf{Подготовка набора тестов}

        \begin{tabular} {|l|l|}
            \hline
                \textbf{Тест} & \textbf{Описание}\\
            \hline
                int a; & Простой тест \\
            \hline
                int *a; & Простой тест №2 \\
            \hline
                int a, b, c; & Тест на правило N $\rightarrow$ ,VN \\
            \hline
                int *a, **b; & Тест на правило V $\rightarrow$ *V \\
            \hline
                int a; int c; & Тест на правило D $\rightarrow$ TKD \\
            \hline
                  & Тест на правило D $\rightarrow$ $\varepsilon$ \\
            \hline
                int *a, b; double c, *d; float e; & Случайный тест \\
            \hline
                int ***a, ******b, ***c; double n, *m; & Случайный тест №2 \\
            \hline
        \end{tabular}


\end{itemize}

\end{document}

%%% Local Variables:
%%% fill-column: 10
%%% mode: latex
%%% coding: utf-8
%%% TeX-PDF-mode: t
%%% End: